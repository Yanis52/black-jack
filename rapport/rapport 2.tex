\documentclass[french,12pt]{article}
\usepackage{babel}

\usepackage[utf8]{inputenc}%encodage des caractères
\usepackage[T1]{fontenc}%encodage de la police
\usepackage{graphicx}
\usepackage{amsmath}
\usepackage{hyperref}
\usepackage{tikz}
\usepackage{listings}
\usepackage[linesnumbered,ruled,french,onelanguage]{algorithm2e} 
\setlength{\parindent}{0pt} 

\title{Rapport Méthode de Conception}
\author{Antonin Montagne, Lou-Anne Gautherie, Nathan Sakkriou, \\
Yanis Habarek}
\date{25 Novembre 2022}

\begin{document}

\maketitle

\begin{figure}[h]
	\begin{center}
		\includegraphics[scale=0.4]{img/fond.png}	
	\end{center}
\end{figure}

\begin{figure}[b]
	\begin{center}
		\includegraphics[scale=0.7]{img/unicaen.png}	
	\end{center}
\end{figure}

\thispagestyle{empty}
\setcounter{page}{0}
\newpage

\tableofcontents
\newpage

\section{Introduction}

\subsection{Plan du rapport}

Nous évoquerons d'abord quels étaient nos objectifs de départ (\ref{objectifs}), puis nous détaillerons les différentes étapes de la création de notre projet avec les rôles de chacuns (\ref{organisation}). Ensuite nous présenterons les éléments techniques utilisées (\ref{elemtech}) dans notre code, ainsi que l'achitecture de ce projet (\ref{architecture}). Finalement nous présenterons certaines expérimentations (\ref{experimentations}) et terminerons par une courte conclusion (\ref{conclusion}).

\subsection{Objectifs du projet} \label{objectifs}

Nous avions pour but de créer un jeu de Blackjack. Certaines contraintes nous étaient données :
\begin{itemize}
	\item Respecter la forme MVC.
	\item Créer une dépendance entre une librairie "\texttt{cartes}" et un jeu "\texttt{blackjack}".	
\end{itemize}

\section{Fonctionnalités implémentées} \label{organisation}

\subsection{Description des fonctionnalités}

A partir de l'accueil, il est d'abord possible de vérifier les règles du blackjack en cliquant sur le bouton "\textsc{règles du jeu}". \\
Le clic du bouton renverra le joueur sur le site \textit{guide-blackjack.com/regles-du-black-jack.html}, qui contient les règles de ce jeu. \\\\
Il est aussi proposé de lancer une partie à l'aide du bouton "\textsc{nouvelle partie}". \\
Le clic du bouton ouvrira un formulaire qui demandera à l'utilisateur, combien de joueur veut-il rentrer pour cette partie. Une fois le nombre de joueur rentré, le jeu se lancera après avoir cliqué sur "\textsc{Commencer}". Le joueur commencera la partie, il peut soit tirer une carte, soit ne rien faire (boutons  "\textsc{Hit}" et "\textsc{Stop}"). Si il choisit de ne rien faire, le tour passe au croupier. Le plus proche de 21 sans dépasser gagne la partie. Une fois la partie terminée, une fenêtre pop-up avec le nom de celui qui a gagné et un bouton "\textsc{OK}" qui renverra le joueur sur la page d'accueil.\\\\
Depuis l'accueil, il est finalement possible de fermer la fenêtre en appuyant sur "\textsc{Cliquer}". 

\subsection{Organisation du projet}

\section{Eléments techiniques} \label{elemtech}

\subsection{Paquetages utilisées}

\subsubsection{Paquetages Java}

\subsubsection{Paquetages JAR}

\subsection{Algorithmes utilisés}

\section{Architecture du projet} \label{architecture}

\section{Expérimentations} \label{experimentations}

\section{Conclusion} \label{conclusion}

\end{document}